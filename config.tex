\usepackage{amsmath,amssymb,array,tikz}
\usetikzlibrary{datavisualization,datavisualization.formats.functions}
\usepackage{caption}
\captionsetup{font={scriptsize}}

\newcommand{\myvec}[1]{\ensuremath{\boldsymbol{#1}}}
\newcommand{\gaussiandist}[2]{
    \frac{1}{\sqrt{2\pi}{#2}}\exp\left(-\frac{(x-#1)^2}{2{#2}^2}\right)
}
\newcommand{\ngaussiandist}[3][n]{
    \frac{1}{(2\pi)^{{#1}/2}|{#3}|^{1/2}}\exp\left(-\frac{1}{2}(\myvec{x}-{#2})^T{#3}^{-1}(\myvec{x}-{#2})\right)
}
\newcommand{\ngaussiandistvar}[4][n]{
    \frac{1}{(2\pi)^{{#1}/2}|{#4}|^{1/2}}\exp\left(-\frac{1}{2}({#2}-{#3})^T{#4}^{-1}({#2}-{#3})\right)
}
\newcommand{\intd}{\mathrm{d}}
\newcommand{\probability}{\ensuremath{\mathbb{P}}}
\newcommand{\expectation}{\ensuremath{\mathbb{E}}}
\newcommand{\myeqref}[1]{式(\ref{#1})}

\def\tikzgaussian#1#2{
    exp(-0.5*(\value{x}-(#1))*(\value{x}-(#1))/((#2)*(#2)))/(sqrt(2*pi)*(#2))
}

\usepackage{algorithm,algorithmicx,algpseudocode}
\floatname{algorithm}{算法}
\renewcommand{\algorithmicrequire}{\textbf{输入:}}  % Use Input in the format of Algorithm  
\renewcommand{\algorithmicensure}{\textbf{输出:}} % Use Output in the format of Algorithm